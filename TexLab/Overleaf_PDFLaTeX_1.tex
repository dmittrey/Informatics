%% !!!! USE PDFLaTeX to compile !!!
\documentclass[a4paper]{article}

%% \usepackage[UTF8]{ctex} % can be used for LuaLaTex
% Chinese characters:
\usepackage{CJKutf8}

\usepackage{indentfirst} % package for indent for the first line of the paragraph in the section
\usepackage{lipsum} % package for default text (= text-"template")

\begin{document}
\pagenumbering{gobble} % exclude page numbering since current page
\begin{center}

{\Huge 
This is Title Page
}
\vspace{10cm} % add indent between lines

\LARGE Hangzhou

\begin{CJK*}{UTF8}{gbsn} % add Chinese characters
文章内容。

文章内容。

基于宏的流行的文本格式化程序
\end{CJK*}

2020
\rm % clear font of the text
\end{center}

\newpage % start new page
\pagenumbering{arabic} % set page number style
\setcounter{page}{2} % set number of the current page

\section{Part 1. Lipsum}

Hello, world! \textbf{We are lucky to study in HDU! =)}

\lipsum[7] % 7th paragraph

\medskip % small indent between lines 

\lipsum[10] % 10th paragraph

\bigskip % big indent between lines

\lipsum[7-10] % Paragraphs 7-10

%https://tex.stackexchange.com/questions/17611/how-does-one-type-chinese-in-latex

\end{document}